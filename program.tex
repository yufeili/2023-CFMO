%!TEX program = xelatex
\documentclass[letter,8pt,poets]{ConcProg}
\usepackage{geometry}
\geometry{scale=0.88}
\usepackage{ctex}
\usepackage{setspace}
\usepackage[utf8]{inputenc}
\usepackage[T1]{fontenc}
\usepackage{times}
\usepackage{lipsum}
\usepackage{multicol}
\usepackage{setspace} % 行间距
% \newCJKfontfamily\nihongo{IPAexMincho}


\setlength{\parindent}{0pt}
%\setmainfont{Times New Roman}


%\setCJKmainfont{NotoSerifCJK} % 或 \setCJKmainfont {KaiTi} 中文字体
\setCJKmainfont{FangSong}
% \newCJKfontfamily\nihongo{HanaMinA}
%\setCJKmainfont{STFangSong}

%\setCJKfamilyfont {zhkai}{simkai} % 华文行楷  
%\newcommand{\zhkai}{\CJKfamily{zhkai}}
%\pagestyle{empty}

\begin{document}
%\pagestyle{empty}
%\begin {spacing}{1.1} %% 行间距

 \begin{spacing}{1.0} 
\begin{center}
\Huge{C}\large{HINESE } \Huge{F}\large{OLK} \Huge{M}\large{USIC} \Huge{O}\large{RCHESTRA} \Huge{C}\LARGE{ONCERT}
\end{center}
\begin{programme}{
\\  {}
}
  \begin{part}[]
    
    
   \begin{composition}{Li Huanzhi (李焕之)}{}{Spring Festival Overture (春节序曲)}{}      
    \end{composition} 
    
    
     \begin{composition}{Lu Lianghui (卢亮辉)}{}{Childhood Reminiscence (童年的回忆)}{}      
    \end{composition} 
    
     \begin{composition}{Nie Er (聂耳)\\ Arranged by Li Junping (黎俊平)}{}{Blooming Flowers and Full Moon (花好月圆)}{}      
    \end{composition} 
    
    \begin{composition}{Luo Maishuo (罗麦朔)}{}{Wengloumi (嗡喽咪)- II. Seeding hemp seeds (撒麻)}{}      
    \end{composition} 
    
    \begin{composition}{Jay Chou (周杰伦)}{}{Orchid Pavilion Preface (兰亭序)}{}      
    \end{composition} 
    
     \begin{composition}{Tan Xuan (谭旋)}{}{Bracing the Chill (凉凉)}{}      
    \end{composition}      
    \\
    $$\textsl{Intermission}$$
    \begin{composition}{\ }{}{Performance from Dragon Phoenix Wushu Team (龙凤武术队)}{}      
    \end{composition} 
     \begin{composition}{\ }{}{Performance from Dance of the Soul from China (舞之魂)}{}      
    \end{composition} 
    \begin{composition}{Qian Lei (钱雷)}{}{Big Fish (大鱼)}{}      
    \end{composition} 
    
    
    
    \begin{composition}{Chen Yu-Peng(陈致逸) \\Arranged by Li Yufei}{}{Lantern Rite Festival in Genshin Impact - Mingxiao Convergence \\(原神海灯节 $-$ 汇成明霄)}{}      
    \end{composition} 
     \begin{composition}{Yokiko Isomura(磯村由紀子)}{}{The Street Where Wind Resides (風の住む街)}{}      
    \end{composition} 
    
     \begin{composition}{Ballad of Minnan(闽南)\\ Arranged by Hu Junxian (胡峻贤)}{}{Tea Picking Ballads (采茶歌)}{}      
    \end{composition} 
    
    \begin{composition}{Weng Qingxi(翁清溪) }{}{The Moon Represents My Heart (月亮代表我的心)}{}      
    \end{composition} 
    
    \begin{composition}{Folk Music of Qinghai	(青海) \\ Arranged by Huang Zhenfen (黄振奋)}{}{Flower and Youth (花儿与少年) }{}      
    \end{composition} 
    
    \begin{composition}{Liu Tieshan(刘铁山), \ \ Mao Yuan (茅沅)}{}{Dance of the Yao People (瑶族舞曲)}{}      
    \end{composition} 
    
     \begin{composition}{Su Wenqing (苏文庆)}{}{Capriccio Taiwan (台湾追想曲)}{}      
    \end{composition} 
    
  
    \end{part}
\end{programme}
\newpage


\begin{center}

\Large\textsl{ Roster}
\end{center}
%\zhkai
\begin{multicols}{2}% 第二段分两栏 \\
\begin{spacing}{1.5}
Conducter:\\
Prof. Shen  Han-wei(沈漢威)
\\
\\
Violin:\\
Deng  Chuyang(邓初阳)\\
\\
Erhu(二胡):     \\   
Yang  Jiaxin(杨嘉欣)\\  
Fan  Maggie(樊美辰)\\
Shi Neng(施能)\\
Yang  Dana (楊詠喬)\\
Zhao  Kan(赵侃)\\
\\
Cello:\\
Li  Yufei(李宇飞)\\

Pipa(琵琶):\\
Zhang  Wenzhuo(张文卓)\\
\\
Guzheng({古筝}):\\
Lu  Shiyu(卢诗妤)\\
Tu  Yamei(涂雅梅)\\
Yu Xiaojia(于晓佳)
\\
\\
Dizi(笛子): 				 \\
Wang  Cathy(王薏慈)\\
Xiong  Tianyu (熊天宇)\\
\\
Suona(唢呐):\\
Xu  Yizi(许易孜)
\\
\\
\\
Percussion:\\
Brickner, Dustin\\
Chen Fatong(陈法同)\\
\end{spacing}
\end{multicols}
\begin{center}
\Large\textsl{ Others}
\end{center}
\begin{multicols}{2}%% 场务
Set Coordinator: Shen Han-Wei, Xiong Tianyu\\ 
Photographer:Qiu Rui(邱瑞)\\
Host: Wang Xiaoqi(王筱淇), Yu Xiaojia, Yang Yufeng(杨宇峰) \\
Audiovisual Manager: Gao Yiyang (高艺洋) \\
Promotion: Tu Yamei, Yang Jiaxin, Cathy Wang, Li Yufei, Yu Xiaojia, Deng Chuyang, Xiong Tianyu, Ren Sida (任思达)


\end{multicols}
\begin{center}
\Large\textsl{ Introduction to CFMO}
\end{center}

CFMO (Chinese Folk Music Orchestra) is an organization at the Ohio State University(tOSU) that is open to all students, staff, and faculty who are interested in Chinese folk music and instruments. We participate in various multi-cultural and educational events in the Columbus Metropolitan area, such as, performing at different cultural events on campus, teaching instruments at Columbus Chinese school and tOSU, street performance at short north, opening and closing performance at East Meet West concert with Columbus Symphony Orchestra and so on. Our mission is to enrich the culture diversity around Columbus area through our music.
\end{spacing}
\end{document}