%!TEX program = xelatex
\documentclass[letter,6pt,poets]{ConcProg}
\usepackage{geometry}
\geometry{scale=0.88}
\usepackage{ctex}
\usepackage[utf8]{inputenc}
\usepackage[T1]{fontenc}
\usepackage{times}
\usepackage{multicol}
\usepackage{setspace} % 行间距
% \newCJKfontfamily\nihongo{IPAexMincho}
%%\newCJKfontfamily\nihongo{Hanazono}

\setlength{\parindent}{0pt}
%\setmainfont{Times New Roman}


%\setCJKmainfont{KaiTi} % 或 \setCJKmainfont {KaiTi} 中文字体
% \setCJKmonofont{KaiTi}

%\setCJKfamilyfont {zhkai}{simkai} % 华文行楷  
%\newcommand{\zhkai}{\CJKfamily{zhkai}}
%\pagestyle{empty}

\begin{document}
%\pagestyle{empty}
%\begin {spacing}{1.1} %% 行间距

 \begin{spacing}{1.0} 
\begin{center}
\Huge{C}\large{HINESE } \Huge{F}\large{OLK} \Huge{M}\large{USIC} \Huge{O}\large{RCHESTRA} \Huge{C}\LARGE{ONCERT}
\end{center}
\begin{programme}{
\\  {\normalsize 9 ~Apr 2023}
}
  \begin{part}[]
    
    
    \begin{composition}{Nie Er (聂耳), Arr. Li Junping (黎俊平)}{}{Blooming Flowers and Full Moon (花好月圆)}{}      
    \end{composition}
    
    
    \begin{composition}{Liu Tieshan (刘铁山), Mao Yuan (茅沅) \\Arr. Peng Xiuwen (彭修文)}{}{Dance of Yao People  (瑶族舞曲)}{}
    \end{composition}
     
    \begin{composition}{Weng Qingxi (翁清溪),\ Arr.\ Li Junping (黎俊平) }{}{The Moon Represents My Heart  (月亮代表我的心)}{}
  
    \end{composition}
    
    
  
  \end{part}
  
  
\end{programme}

\begin{center}
\Large\textsl{ Roster}
\end{center}
%\zhkai
\begin{multicols}{2}% 第二段分两栏 \\
Conducter: Su Xin  (苏昕)
\\
\\
 Erhu(二胡):     \\     
Zhao Kan (赵侃)\\
\\
Pipa(琵琶):\\
\\
Cello:\\
Li Yufei (李宇飞)\\
\\
\\
\\
\\
\\
Dizi(笛子): 				 \\
Xiong Kyle (熊天宇)\\
\\
Guzheng({古筝}):\\
\\
Percussion:\\
\\
\\
\\
Special Guest:\\
??? 
 \footnotesize{ from } \emph{ ??}
\\
\end{multicols}
\begin{center}
\Large\textsl{ Others}
\end{center}
\begin{multicols}{2}%% 场务
 Set Coordinator: ?,\\ 
 Video editing:? \\
Photographer:?\\
AV Manager: ?\\
Promotion:?\\
\\


\end{multicols}
\begin{center}
\Large\textsl{ Introduction to CFMO}
\end{center}

CFMO (Chinese Folk Music Orchestra) is an organization at the Ohio State University that is open to all students, staff, and faculty who are interested in Chinese folk music and instruments. We participate in various multi-cultural and educational events in the Columbus Metropolitan area, such as, performing at different cultural events on campus, teaching instruments at Columbus Chinese school and OSU, street performance at short north, opening and closing performance at East Meet West concert with Columbus Symphony Orchestra and so on. Our mission is to enrich the culture diversity around Columbus area through our music.
\end{spacing}
\end{document}